
\documentclass{beamer}

\usetheme{Edinburgh}

\titlegraphic{%
    \begin{center}
    \begin{minipage}{0.3\textwidth}    
        \centering
        \includegraphics[height=3cm,keepaspectratio]{atlas.png}
    \end{minipage}
    \begin{minipage}{0.3\textwidth}    
        \centering
        \includegraphics[height=3cm, keepaspectratio]{LOGO-ERC_negatif.pdf}
    \end{minipage}
    \begin{minipage}{0.3\textwidth}    
        \centering
        \includegraphics[width=3cm,height=3cm,keepaspectratio]{UoE.png}
    \end{minipage}
    \end{center}
}

% Title page details: 
\title{Custom University of Edinburgh Theme} 
\subtitle{BeamerCon 2077}
\author{Alex Veltman}
\date{\today}
\institute[UoE]{University of Edinburgh}

\hypersetup{
  colorlinks,
  allcolors=.,
  urlcolor=beamer@blue-bright,
}

\begin{document}

% Title page frame
\begin{frame}[plain]
    \titlepage 
\end{frame}

% Remove logo from the next slides
\logo{}

% Lists frame
\begin{frame}{Lists in Beamer}

This is an unordered list:
\begin{itemize}
    \item Item 1
    \item Item 2
    \item Item 3
\end{itemize}

and this is an ordered list:
\begin{enumerate}
    \item Item 1
    \item Item 2
    \item Item 3
\end{enumerate}

\end{frame}

\begin{frame}{Mathematics}
    Maxwell's equations in their differential form are:

    \begin{align}
        \nabla \cdot \mathbf{E} + E&= \frac{\rho}{\varepsilon_0} \\
        \nabla \cdot \mathbf{B} &= 0 \\
        \nabla \times \mathbf{E} &= -\frac{\partial \mathbf{B}}{\partial t} \\
        \nabla \times \mathbf{B} &= \mu_0 \mathbf{J} + \mu_0 \varepsilon_0 \frac{\partial \mathbf{E}}{\partial t}
    \end{align}
    
    where:
    \begin{itemize}
        \item $\mathbf{E}$ is the electric field,
        \item $\mathbf{B}$ is the magnetic field,
        \item $\rho$ is the charge density,
        \item $\mathbf{J}$ is the current density,
        \item $\varepsilon_0$ is the vacuum permittivity, and
        \item $\mu_0$ is the vacuum permeability.
    \end{itemize}
\end{frame}

\begin{frame}{Blocks in Beamer}
    \begin{block}{Standard Block}
        This is a standard block.
    \end{block}
    \begin{alertblock}{Alert Message}
        This block presents alert message.
    \end{alertblock}
    \begin{exampleblock}{An example of typesetting tool}
        Example: MS Word, \LaTeX{}
    \end{exampleblock}
\end{frame} 

\end{document}